One thing we can discover is that human language can be subject to statistical regularities. 

One of the most studied statistical laws which affect linguistics is called the Zipf's law, a common phenomenon that seems to occur in complex systems where discrete units self-organize into groups, or types \cite{Corral2020}. The law of abbreviation draws a relation between the frequency of appearance of a word and its length, as it generally states that the more frequent a word is, the shortest it tends to be. This is known as the ``Principle of least effort", which was hypothesized as the result from pressure of having both: accurate and efficient communication in language.

As languages overall have a finite inventory that can be recombined into words, to satisfy these pressures languages assign shorter words to most frequent meanings and longer words to the less frequent.

For this task, we aim to determine if this law holds for languages of different families by measuring the lengths of the words, degree of optimality and correlation between the length and the frequency of words.