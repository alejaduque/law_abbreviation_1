In terms of the length of the words, we see that for all of the languages we tested the average length tends to be between 1 and 8 characters. The minimum length of words tends to be one to two characters less than the mean length of words for each of the languages, meaning that the minimum and the average length of the words in all observed cases tend to be proportional.

In \cref{plot2}, we see how the languages that are more compressed are the ones which have in common the fact that they come from different parts of Asia. The most compressed of them seems to be the case of Chinese, which could be explained by the fact that Chinese is a monosyllabic language; meaning that the words in this language have complex structures which are able to represent whole concepts \cite{michaud2012monosyllabicization}. However, for the case of Japanese and Korean, which are polysyllabic languages (i.e. words can have multiple syllables), according to our results they might reflect a tendency to have words that consist of monosyllables in its system. 
In contrast, within the least compressed languages, we see languages like Kannada, which seem to have a tendency for words to be larger when compared to the rest of observed languages.  What this tells is that in monosyllabic languages as Chinese, the system allows the information of words to be compressed in a single complex character, in comparison to least compressed languages, where information is encoded in longer strings of characters. 

When looking at optimization data in the results, we see that the degree of optimality values obtained ranged between the scores of 0.6 and 1. On the other hand, when measuring the optimality scores, we saw these values ranged from 0.3 to 1. 

In \cite{FerrerICancho2018} it is mentioned that real languages are optimized in an average of 30\%. However, we find that for our parallel corpus results reflected much higher values which ranged between 60\% to 100\% in coding efficiency. We could assume the reason why these values are higher is due to the size of our corpus, as for this task we used a relatively short text that contained a variety of types. Furthermore, this text is of a specific genre and does not represent the language as a whole.

Regarding the correlation, we found very small p-values (smaller than 0.05 in all cases) and, therefore, we did find correlation between frequency and length of tokens. All the tau values are negative with different values between -0.3 and -0.1, which means that there is a negative association between frequency and length; therefore the bigger the frequency, the smaller the length. This results are aligned with the law of abbreviation. While all results showed a strong correlation, the tau values did differ, which means that some languages have a more negative correlation than others.
