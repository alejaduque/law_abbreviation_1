In order to test whether the law of abbreviation holds in different languages, we picked 20 languages from 11 different families \footnote{We used the Glottolog database \cite{Glottolog} for collecting these languages. More specifically: 6 Indo-european, 3 Austronesian, 2 Atlantic-congo, 2 Uralic, 1 Afro-asiatic, 1 Dravian, 1 Japonic, 1 Koreanic, 1 Sino-tibetan, 1 Turkic and 1 non-determined family (as for the case of Basque language).} for which we collected a parallel corpus.  By using the NLTK library, we collected a parallel corpus with samples of the Universal Declaration of Human Rights (UDHR) for each of the languages we wanted to study. We used a parallel corpus so we could see the difference in results within different codes when treating with text of the same origin and context. 

\newpage
We used the Python for gathering and pre-processing the texts and R for statistical analysis. Specifically, we used the spaCy library to tokenize the texts and obtain a table with token, length and frequency columns for each language. During this pre-processing, we also discarded all of the noisy data that resulted from this segmentation (i.e. punctuation marks, spaces and numerical data).
